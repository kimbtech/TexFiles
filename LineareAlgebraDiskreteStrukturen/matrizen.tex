\documentclass{scrartcl}

\renewcommand*{\familydefault}{\sfdefault}
\usepackage[T1]{fontenc}
\usepackage{lmodern}
\usepackage[utf8]{inputenc}

\usepackage[ngerman]{babel} % neue deutsche Trennungsregeln, etc

\usepackage[scaled=1]{helvet} %Schriftart

\usepackage{graphicx, color}
\usepackage{amssymb, amsfonts, amsthm, amsmath, float}
\usepackage{numprint}
\usepackage{stmaryrd}

\usepackage{fancyhdr}
\pagestyle{empty} 

\begin{document}

	\section*{Matrizen}
	
	\subsection*{Schreibweisen von Matrizen}
		\begin{align*}
			&A \in \mathbb{K}^{n,p},
			B \in \mathbb{K}^{p,m},
			C \in \mathbb{K}^{n,m}
			\\
			&A = (a_{ij})_{i=1,...,m ~~ j=1,...,p}
			\\
			&(A)_{ij} = a_{ij} = A_{(ij)}
			\\
			&C := A \cdot B
				= (c_{jk})_{j=1,...,n ~~ k=1,...,m}
				= \left( \sum\limits_{i=1}^{p} a_{ji} b_{i,k} \right)_{j=1,...,n ~~ k=1,...,m}
			\\
			&&c_{jk}
				= \sum\limits_{i=1}^{p} a_{ji} b_{i,k}
		\end{align*}
	
	\subsection*{Rang}	
		\[
			ZR(A) := spann\left( A(j,:), j=1,...,n \right)
		\] 
		\[
			SR(A) := spann\left( A(:,j), j=1,...,m \right)
		\] 
		$Rang(A) = dim(ZR(A)) = dim(SR(A))$

		\subsubsection*{Dimensionsformel}
			$A \in \mathbb{K}^{n,n}$
			\[
				\underbrace{ dim( Kern(A) ) }_{\text{Anzahl Freiheitsgrade}} + Rang(A) = n
			\]	
		
		{\tiny
			\today
		}
\end{document}