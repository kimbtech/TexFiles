\documentclass{kimbblaetter}

%Aufgabenblatt und Personeninfo
\defineFach{Fach}
\defineAuthor{Max Mustermann}
\defineStudents{Max Mustermann 00000 - Otto Müller 000000}
\defineGruppe{Gruppe 12}
\defineNummer{ 5 }

%Ausgabemodi
%	Namen nach oben links?
%\rearrangeUp
%	Finale Ausgabe?
%\outputFinal
%	Spaetere Weitergabe?
%\laterPublish

\begin{document}
	%Klasseninfo im Draftmode
	\showinfo

	%Nutzung der Klasse
	\aufgabe{Übung 2}
	\kommentar{Test}
	
	\aufgabe{Task 3}
	\showbeta{ $5 \cdot 5 = 24 $ }
	\fixed{Das war falsch, $25$ wäre korrekt.}
	
	\aufgabe{Bonus}
	Und jetzt ein schöner Beweis!
	\beweisbox
	\\
	Das stimmt nicht $ 1 = 3 $ \kreuzf, aber das $ 1 < 3 $\hakenw!
	\\
	\\
	\defbox{red}{
		\textbf{ Wichtiges in einer schönen Aufzählung }\\
		\begin{enumerate}[(W1)]
			\item
				Nummer 1
			\item
				Nummer 2
		\end{enumerate}
	}	
	
	\aufgabe{ Quellcode }
	\begin{lstlisting}
		/**
		* Diese Methode addiert!
		* @param a Ein Summand
		* @param b Weiterer Summand
		* @return Summe
		*/
		public int addiere( int a, int b ){
			return a + b;
		}
		private void helpMe(){
			System.out.println( "F1, F1, ..." );
		}
	\end{lstlisting}
	
\end{document}
