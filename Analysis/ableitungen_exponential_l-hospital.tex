\documentclass{scrartcl}

\renewcommand*{\familydefault}{\sfdefault}
\usepackage[T1]{fontenc}
\usepackage{lmodern}
\usepackage[utf8]{inputenc}

\usepackage[ngerman]{babel} % neue deutsche Trennungsregeln, etc

\usepackage[scaled=1]{helvet} %Schriftart

\usepackage{graphicx, color}
\usepackage{amssymb, amsfonts, amsthm, amsmath, float}
\usepackage{numprint}
\usepackage{stmaryrd}

\usepackage{fancyhdr}
\pagestyle{empty} 

\begin{document}

	\section*{Exponentialfunktionen}
	\subsection*{Vereinfachte Schreibweise}

	\[
	e^x =: exp( x )
	\]

	\subsection*{Ableitungsregeln}

	Sei $I \subseteq \mathbb{R}$ ein Intervall und seien $b: I \to (0, \infty)$ und $r: I \to \mathbb{R}$ differenzierbare
	Funktionen.\\
	Durch \[
		f(x) = b(x)^{r(x)} ~~~~~~~, x \in I
	\] wird eine Funktion $f: I \to (0, \infty) \to \mathbb{R}$ definiert.\\
	Es gilt die Darstellung:
	\[
		f(x) = exp( r(x) \ln(b(x)))
	\]
	also folgt mit der Ketten- und Produktregel die Differenzierbarkeit und 
	\[
		f'(x) = exp( r(x) \ln(b(x)) )\left( r'(x) \ln(b(x)) + r(x) \frac{b'(x)}{b(x)} \right)
	\]
	
	\begin{itemize}
		\item
			für r konstant und $b(x) = x$\\
			\[
				f(x) = x^r = exp( r \ln(x) )
			\]
			\[
				f'(x) = exp( r \ln(x) ) \left( \frac{r}{x} \right) = x^r \frac{r}{x} = r x^{(r-1)}
			\]
		\item
			Für $b > 0$ und $r(x) = x$\\
			\[
				f(x) = b^x = exp( x \ln( b ) )
			\]
			\[
				f'(x) = exp( x \ln( b ) )  \ln(b) = b^x \ln(b)
			\]
			
	\end{itemize}
	
	\section*{Regeln von de l'Hospital}
	
	\subsection*{Satz}
	
	Seien $f,g$ definiert, differenzierbar und $g'(x) \neq 0$ für alle
	$x \neq a$ in einer Umgebung von $a \in \mathbb{R} \cup \{ -\infty, \infty\}$.
	In jeder der beiden Situation 
	\begin{enumerate}
		\item
			$f(x) \to 0$ und $g(x) \to 0$ für $x \to a$
		\item			
			$f(x) \to \pm \infty $ und $ g(x) \to \pm \infty$ für $x \to a$
	\end{enumerate}
	gilt dann:
	\[
		\exists \lim\limits_{x \to a} \frac{f'(x)}{g'(x)} \in \mathbb{R} \cup \{ -\infty, \infty\}
		\Rightarrow
		\lim\limits_{x \to a} \frac{f'(x)}{g'(x)} = \lim\limits_{x \to a} \frac{f(x)}{g(x)}
	\]
	
	\subsection*{Warnungen}
		\begin{itemize}
			\item
				Aus der Existenz von $ \lim\limits_{x \to a} \frac{f'(x)}{g'(x)} $ darf unter der Voraussetzung
				\glqq$\frac{0}{0}$\grqq ~ auf die Existenz von $ \lim\limits_{x \to a} \frac{f(x)}{g(x)} $ geschlossen werden.\\
				Umgekehrt kann aber $ \lim\limits_{x \to a} \frac{f(x)}{g(x)} $ existieren, obwohl
				$ \lim\limits_{x \to a} \frac{f'(x)}{g'(x)} $ \underline{nicht} existiert. \\
				
				\begin{description}
					\item[Beispiel]
					$f(x) = x^2 \cos(\frac{1}{x}), ~~~ g(x) = x, ~~~ a = 0$\\
					\[
						\lim\limits_{x \to 0} \frac{f'(x)}{g'(x)} = \lim\limits_{x \to 0} f'(x) 
						~~~~~~ \text{ existiert nicht}
					\]
					Aber $\lim\limits_{x \to 0} x^2 \cos(\frac{1}{x}) = 0$
				\end{description}
				
			\item
				Die Voraussetzung \glqq$\frac{0}{0}$\grqq ~ bzw. \glqq$\frac{\infty}{\infty}$\grqq ~
				muss bestehen und \underline{jedes Mal} geprüft werden!\\
			
				\begin{description}
					\item[Beispiel]
						{\color{red}
						\[
							\lim\limits_{ x \to 0} \frac{\sin(x)}{x^2} = 
							\underbrace{\lim\limits_{ x \to 0} \frac{\cos(x)}{2x}}_{ \text{nicht } \glqq\frac{0}{0}\grqq } =
							\underbrace{\lim\limits_{ x \to 0} \frac{- \sin(x)}{2}}_{ \text{nicht } \glqq\frac{0}{0}\grqq } = 0
						\]
						{\tiny obiges ist falsch !!}
						}
						
				\end{description}	
			
		\end{itemize}

		{\tiny
			Version 1.0 - \today
		}
\end{document}